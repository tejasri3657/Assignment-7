\documentclass[journal,12pt,twocolumn]{IEEEtran}

\usepackage{setspace}
\usepackage{gensymb}

\singlespacing


\usepackage[cmex10]{amsmath}

\usepackage{amsthm}

\usepackage{mathrsfs}
\usepackage{txfonts}
\usepackage{stfloats}
\usepackage{bm}
\usepackage{cite}
\usepackage{cases}
\usepackage{subfig}

\usepackage{longtable}
\usepackage{multirow}

\usepackage{enumitem}
\usepackage{mathtools}
\usepackage{steinmetz}
\usepackage{tikz}
\usepackage{circuitikz}
\usepackage{verbatim}
\usepackage{tfrupee}
\usepackage[breaklinks=true]{hyperref}
\usepackage{graphicx}
\usepackage{tkz-euclide}
\usepackage{float}

\usetikzlibrary{calc,math}
\usepackage{listings}
    \usepackage{color}                                            %%
    \usepackage{array}                                            %%
    \usepackage{longtable}                                        %%
    \usepackage{calc}                                             %%
    \usepackage{multirow}                                         %%
    \usepackage{hhline}                                           %%
    \usepackage{ifthen}                                           %%
    \usepackage{lscape}     
\usepackage{multicol}
\usepackage{chngcntr}

\DeclareMathOperator*{\Res}{Res}

\renewcommand\thesection{\arabic{section}}
\renewcommand\thesubsection{\thesection.\arabic{subsection}}
\renewcommand\thesubsubsection{\thesubsection.\arabic{subsubsection}}

\renewcommand\thesectiondis{\arabic{section}}
\renewcommand\thesubsectiondis{\thesectiondis.\arabic{subsection}}
\renewcommand\thesubsubsectiondis{\thesubsectiondis.\arabic{subsubsection}}


\hyphenation{op-tical net-works semi-conduc-tor}
\def\inputGnumericTable{}                                 %%

\lstset{
%language=C,
frame=single, 
breaklines=true,
columns=fullflexible
}
\begin{document}
\newtheorem{theorem}{Theorem}[section]
\newtheorem{problem}{Problem}
\newtheorem{proposition}{Proposition}[section]
\newtheorem{lemma}{Lemma}[section]
\newtheorem{corollary}[theorem]{Corollary}
\newtheorem{example}{Example}[section]
\newtheorem{definition}[problem]{Definition}

\newcommand{\BEQA}{\begin{eqnarray}}
\newcommand{\EEQA}{\end{eqnarray}}
\newcommand{\define}{\stackrel{\triangle}{=}}
\bibliographystyle{IEEEtran}
\providecommand{\mbf}{\mathbf}
\providecommand{\pr}[1]{\ensuremath{\Pr\left(#1\right)}}
\providecommand{\qfunc}[1]{\ensuremath{Q\left(#1\right)}}
\providecommand{\sbrak}[1]{\ensuremath{{}\left[#1\right]}}
\providecommand{\lsbrak}[1]{\ensuremath{{}\left[#1\right.}}
\providecommand{\rsbrak}[1]{\ensuremath{{}\left.#1\right]}}
\providecommand{\brak}[1]{\ensuremath{\left(#1\right)}}
\providecommand{\lbrak}[1]{\ensuremath{\left(#1\right.}}
\providecommand{\rbrak}[1]{\ensuremath{\left.#1\right)}}
\providecommand{\cbrak}[1]{\ensuremath{\left\{#1\right\}}}
\providecommand{\lcbrak}[1]{\ensuremath{\left\{#1\right.}}
\providecommand{\rcbrak}[1]{\ensuremath{\left.#1\right\}}}
\theoremstyle{remark}
\newtheorem{rem}{Remark}
\newcommand{\sgn}{\mathop{\mathrm{sgn}}}
\providecommand{\abs}[1]{\left\vert#1\right\vert}
\providecommand{\res}[1]{\Res\displaylimits_{#1}} 
\providecommand{\norm}[1]{\left\lVert#1\right\rVert}
%\providecommand{\norm}[1]{\lVert#1\rVert}
\providecommand{\mtx}[1]{\mathbf{#1}}
\providecommand{\mean}[1]{E\left[ #1 \right]}
\providecommand{\fourier}{\overset{\mathcal{F}}{ \rightleftharpoons}}
%\providecommand{\hilbert}{\overset{\mathcal{H}}{ \rightleftharpoons}}
\providecommand{\system}{\overset{\mathcal{H}}{ \longleftrightarrow}}
	%\newcommand{\solution}[2]{\textbf{Solution:}{#1}}
\newcommand{\solution}{\noindent \textbf{Solution: }}
\newcommand{\cosec}{\,\text{cosec}\,}
\providecommand{\dec}[2]{\ensuremath{\overset{#1}{\underset{#2}{\gtrless}}}}
\newcommand{\myvec}[1]{\ensuremath{\begin{pmatrix}#1\end{pmatrix}}}
\newcommand{\mydet}[1]{\ensuremath{\begin{vmatrix}#1\end{vmatrix}}}
\numberwithin{equation}{subsection}
\makeatletter
\@addtoreset{figure}{problem}
\makeatother
\let\StandardTheFigure\thefigure
\let\vec\mathbf
\renewcommand{\thefigure}{\theproblem}
\def\putbox#1#2#3{\makebox[0in][l]{\makebox[#1][l]{}\raisebox{\baselineskip}[0in][0in]{\raisebox{#2}[0in][0in]{#3}}}}
     \def\rightbox#1{\makebox[0in][r]{#1}}
     \def\centbox#1{\makebox[0in]{#1}}
     \def\topbox#1{\raisebox{-\baselineskip}[0in][0in]{#1}}
     \def\midbox#1{\raisebox{-0.5\baselineskip}[0in][0in]{#1}}
\vspace{3cm}
\title{ASSIGNMENT 7}
\author{A.Tejasri}
\maketitle
\newpage
\bigskip
\renewcommand{\thefigure}{\theenumi}
\renewcommand{\thetable}{\theenumi}
Download all python codes from 
\begin{lstlisting}
https://github.com/tejasri3657/Assignment-7/blob/main/Assignment-7.py
\end{lstlisting}
%
Latex-tikz codes from 
%
\begin{lstlisting}
https://github.com/tejasri3657/Assignment-7/tree/main
\end{lstlisting}
%
\section{Question No 2.38 (a)}
 Find the coordinates of the foci and the vertices,
the eccentricity,the length of the latus
rectum of the hyperbola $\vec{x}^{\top}\myvec{\frac{1}{9} & 0 \\ 0 & \frac{-1}{16}}\vec{x} = 1$.
%
\section{Solution}
\begin{lemma}
The standard form of a conic is given by
\begin{align}
\frac{\vec{y}^{\top}D\vec{y}}{\vec{u}^{\top}\vec{V}^{-1}\vec{u}-f}&=1
\end{align}
\end{lemma}
Given
\begin{align}
\vec{x}^{\top}\myvec{\frac{1}{9} & 0 \\ 0 &\frac{-1}{16} }\vec{x}=1  
\end{align}
we have,
\begin{align}
    \vec{V} = \myvec{\frac{1}{9} & 0 \\ 0 &\frac{-1}{16} }
    \\
    \vec{u}^{\top}\vec{V}^{-1}\vec{u}-f = 1
    \\
    \vec{c} = -\vec{V}^{-1}\vec{u}=\myvec{0 \\ 0}
    \\
    \lambda_1 =  \frac{1}{9}, \lambda_2 = \frac{-1}{16}
\end{align}
Axes of hyperbola is given by
\begin{align}
    \sqrt{\frac{\vec{u}^{\top}\vec{V}^{-1}\vec{u}-f}{\lambda_1}} = 4\\ \sqrt{\frac{f-\vec{u}^{\top}\vec{V}^{-1}\vec{u}}{\lambda_2}} = 3
\end{align}
The vertices are given as
\begin{align}
    \pm\myvec{4 \\ 0}
\end{align}
Coordinates of foci are given by,
\begin{align}
  \vec{F} =\pm\brak{\sqrt{\frac{(\vec{u}^T\vec{V}^{-1}\vec{u}-f)(\lambda_2-\lambda_1)}{\lambda_1\lambda_2}}}\vec{p_1} \label{eq:1}
\end{align}
where, $\vec{p_1} = \myvec{1 \\ 0}$ since the equation of hyperbola is in standard form.
Substituting the values in \eqref{eq:1} we have,
\begin{align}
    \vec{F} = \pm\myvec{ 5 \\ 0}.
\end{align}
Eccentricity of the hyperbola is given by,
\begin{align}
   e = \frac{\sqrt{\frac{(\vec{u}^{\top}\vec{V}^{-1}\vec{u})(\lambda_2-\lambda_1)}{\lambda_1\lambda_2}}}{\sqrt{\frac{\vec{u}^{\top}\vec{V}^{-1}\vec{u}-f}{\lambda_1}}} \label{eq:2}
\end{align}
substituting the values in \eqref{eq:2},we have
\begin{align}
   e = \frac{5}{3}.
\end{align}
Length of the latus rectum is given by,
\begin{align}
    l = \frac{2\brak{{\sqrt{\frac{f-\vec{u}^{\top}\vec{V}^{-1}\vec{u}}{\lambda_2}}}}^2}{\sqrt{\frac{\vec{u}^{\top}\vec{V}^{-1}\vec{u}-f}{\lambda_1}}} \label{eq:3}
\end{align}
substituting the values in \eqref{eq:3},we have
\begin{align}
   l = \frac{32}{3}
\end{align}
Plot of the hyperbola:
\numberwithin{figure}{section}
\begin{figure}[H]
    \centering
    \includegraphics[width=\columnwidth]{hyperbola.png}
    \caption{Hyperbola}
    \label{fig:hyperbola}
\end{figure}  
\end{document}
